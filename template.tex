%%「論文」,「レター」,「レター(C分冊)」,「技術研究報告」などのテンプレート
%% v3.4 [2023/09/12]
%% 1. 「論文」
\documentclass[paper]{ieicej}
%\documentclass[invited]{ieicej}% 招待論文
%\documentclass[survey]{ieicej}% サーベイ論文
%\documentclass[comment]{ieicej}% 解説論文
%\usepackage[dvipdfmx]{graphicx,xcolor}
%%\usepackage[dvips]{graphicx}
\usepackage[fleqn]{amsmath}
%\usepackage{amsthm}
\usepackage{newtxtext}% 英数字フォントの設定を変更しないでください
\usepackage[varg]{newtxmath}% % 英数字フォントの設定を変更しないでください
%\usepackage{amssymb}
%\usepackage{bm}

\setcounter{page}{1}

\field{}
\jtitle{薄膜FSSシートを用いた5GHz帯WiFi電波の遮断について}
\etitle{Study of Blocking of 5GHz WiFi Wave by thin FSS Sheet}
\authorlist{%
 \authorentry{朝田陸斗}{Asada Rikuto}{中大}\MembershipNumber{}
 %\authorentry{和文著者名}{英文著者名}{所属ラベル}\MembershipNumber{}
 %\authorentry[メールアドレス]{和文著者名}{英文著者名}{所属ラベル}\MembershipNumber{}
 %\authorentry{和文著者名}{英文著者名}{所属ラベル}[現在の所属ラベル]\MembershipNumber{}
}
\affiliate[]{}{}
%\affiliate[所属ラベル]{和文所属}{英文所属}
%\paffiliate[]{}
%\paffiliate[現在の所属ラベル]{和文所属}
\jalcdoi{???????????}% ← このままにしておいてください

\begin{document}
\begin{abstract}
\\
%和文あらまし 500字以内
周波数選択表面(FSS)は特定の周波数帯の電波を遮断・透過するデバイスであり、
本研究では5GHz帯のWIFI電波を遮断するFSSの設計を目指し、視認性も考慮して
薄い透明シート上に金属模様を付着して製作することで、実用性の向上を図っている。
\end{abstract}
\begin{keyword}
%和文キーワード 4〜5語
周波数選択表面
\end{keyword}
\begin{eabstract}
%英文アブストラクト 100 words
bora bora bora
\end{eabstract}
\begin{ekeyword}
%英文キーワード
Frequency Slective Surface
\end{ekeyword}
\maketitle

\section{まえがき}
ここからが本文である\\
以下はwikpediaのコピペ
インターネット嫌いを公言しており、『インターステラー』にはパソコン、携帯電話などインターネットを想起させるものは出さなかった。その理由として「ネットのせいでみんな本を読まなくなった。書物は知識の歴史的な体系だ。ネットのつまみ食いの知識ではコンテクストが失われてしまう」[9]と語っている。
『ダークナイト』ではCGではない本物のビルを丸ごと1棟爆破して撮影を行った。『インターステラー』で使われている一部の地球の映像はCGではなく実際にジェット機の先端にIMAXカメラを搭載し成層圏で撮ったものである[10]。また、『TENET テネット』では退役した飛行機(ボーイング747)を購入し、実際に倉庫に激突させて撮影をした。大掛かりな撮影が困難な時はミニチュアなどによる特撮を起用し極力CGの使用を避けている。とはいえ背景合成やワイヤー消しといったデジタル画像処理は欠かせず、本来CGI主体のプロダクションであるDNEGとニューディール・スタジオにアカデミー視覚効果賞3つをもたらしている。
撮影現場では第二班(本編撮影とは別に、背景やアクションシーンなど、ドラマシーケンス間を構成する、つなぎのシーケンスを担当する撮影チーム)監督をほとんど使わず、自らカメラの横に立って撮影を行う姿勢を貫いている。
同じ俳優を積極的に起用することで知られる。特にマイケル・ケインは『バットマン ビギンズ』以降8作品に続けて出演した(『ダンケルク』はカメオ出演、『オッペンハイマー』には出演せず)。
主要な製作スタッフを固定することでも知られ、『プレステージ』以降の監督作全てで音響設計を担当(デヴィッド・フィンチャー作品におけるレン・クライス同様に)しているリチャード・キングは、アカデミー音響編集賞をノーラン作品で3回受賞している。音楽をハンス・ジマー(『バットマン ビギンズ』以降の監督作6作品)、編集をリー・スミス(『バットマン ビギンズ』から7作品)、美術をネイサン・クロウリー(『バットマン ビギンズ』から6作品)が担当。『TENET テネット』と『オッペンハイマー』では音楽をルドウィグ・ゴランソン、編集をジェニファー・レイムに依頼している。撮影は『メメント』以降『ダークナイト ライジング』まで一貫してウォーリー・フィスターを起用していたが、フィスターが映画監督を志向したため、『インターステラー』以降はオランダの撮影監督であるホイテ・ヴァン・ホイテマを起用。2018年ワーナーの依頼で『2001年宇宙の旅』の修復プロジェクトを主導した際にもホイテマを引き入れた。
IMAXを初めて長編映画で使用した監督である。最初の作品は『ダークナイト』。IMAX以外のラージフォーマットと35mmフィルムのシネマスコープ、スーパー35も併用されており、以降作品はみな一貫した画面アスペクト比を持たず、映画館やホームメディアの条件によって複数の画郭が切り替わる。
現在の映画界ではほとんどの監督がデジタルカメラで撮影しているが、彼はフィルムを使った撮影を行っている。2014年8月には、他の数人の映画監督と共に映画スタジオに働きかけ、フィルムメーカーのコダックから今後 一定量のフィルムを購入する契約を締結させたため、経営難だったコダックはフィルム製造の継続が可能になった[11]。
音響面では「無限音階(シェパード・トーン)」を多用していて、ほぼ全作品で使われている。
超大作ながらも脚本はオリジナルであることが多く、作家主義と大作主義の両立に最も成功している一人と評される[12]。
『007』シリーズのファンであり、2010年の『インセプション』公開時に初めて「いつかボンド映画を監督したい」と発言しており、現在もシリーズのプロデューサーと話し合いを続けている。特に『女王陛下の007』が気に入っていると述べている[13]。また、『バットマン』シリーズや『インセプション』がボンド映画の影響を受けていることも明かしている[14]。『バットマン』3部作を監督するにあたって最も影響を受けた映画として、リチャード・ドナー監督の『スーパーマン』と「007」シリーズ、特に『007 ロシアより愛をこめて』を挙げ[15]、『ダークナイト』ではヒース・レジャー演じるジョーカーが『ロシアより愛をこめて』に登場するナイフ付きの靴を使用するシーンがある。また『私を愛したスパイ』以来「007」シリーズでフィジカル・エフェクトやミニチュア撮影を担当しているクリス・コーボールドを特技監督に起用し、『007 サンダーボール作戦』末尾のフルトン回収システムを『ダークナイト』に「スカイフック」として登場させ、『消されたライセンス』冒頭の飛行機を飛行機で釣り上げる場面は『ダークナイト ライジング』でそっくりな見せ場を作っている。
2013年には「Sight and Sound マガジン」にて、好きな映画として『殺し屋たちの挽歌』(1984年)、『十二人の怒れる男』(1957年)、『シン・レッド・ライン』(1998年)、『怪人マブゼ博士』(1933年)、『ジェラシー』(1980年)、『戦場のメリークリスマス』(1983年)、『宇宙へのフロンティア』(1989年)、『コヤニスカッツィ』(1983年)、『アーカディン/秘密調査報告書』(1955年)、『グリード』(1925年)の10本を挙げている[16]。

\ack %% 謝辞
\\
感謝する

%\bibliographystyle{sieicej}
%\bibliography{myrefs}
\begin{thebibliography}{99}% 文献数が10未満の時 {9}
\bibitem{}
\end{thebibliography}

\appendix
\section{}

%% 著者紹介・顔写真の掲載はC分冊の場合は任意です.
\begin{biography}
\profile{}{}{}
%\profile{会員種別}{名前}{紹介文}% 顔写真あり
%\profile*{会員種別}{名前}{紹介文}% 顔写真なし
\end{biography}

\end{document}



%% 2. 「レター」
\documentclass[letter]{ieicej}
%\usepackage[dvipdfmx]{graphicx,xcolor}
%%\usepackage[dvips]{graphicx}
\usepackage[fleqn]{amsmath}
%\usepackage{amsthm}
\usepackage{newtxtext}% 英数字フォントの設定を変更しないでください
\usepackage[varg]{newtxmath}% % 英数字フォントの設定を変更しないでください
%\usepackage{amssymb}
%\usepackage{bm}

\setcounter{page}{1}

\typeofletter{研究速報}
%\typeofletter{紙上討論}
%\typeofletter{問題提起}
%\typeofletter{ショートノート}
\field{}
\jtitle{}
\etitle{}
\authorlist{%
 \authorentry{}{}{}{}\MembershipNumber{}
 %\authorentry{和文著者名}{英文著者名}{会員種別}{所属ラベル}\MembershipNumber{}
 %\authorentry{和文著者名}{英文著者名}{会員種別}{所属ラベル}[現在の所属ラベル]\MembershipNumber{}
}
\affiliate[]{}{}
%\affiliate[所属ラベル]{和文所属}{英文所属}
%\paffiliate[]{}
%\paffiliate[現在の所属ラベル]{和文所属}
\jalcdoi{???????????}% ← このままにしておいてください

\begin{document}
\maketitle
\begin{abstract}
%和文あらまし 120字以内
\end{abstract}
\begin{keyword}
%和文キーワード 4〜5語
\end{keyword}
\begin{eabstract}
%英文アブストラクト 50 words
\end{eabstract}
\begin{ekeyword}
%英文キーワード
\end{ekeyword}

\section{まえがき}


\ack %% 謝辞

%\bibliographystyle{sieicej}
%\bibliography{myrefs}
\begin{thebibliography}{99}% 文献数が10未満の時 {9}
\bibitem{}
\end{thebibliography}

\appendix
\section{}

\end{document}



%% 3. 「レター(C分冊)」
\documentclass[electronicsletter]{ieicej}
%\usepackage[dvipdfmx]{graphicx,xcolor}
%%\usepackage[dvips]{graphicx}
\usepackage[fleqn]{amsmath}
%\usepackage{amsthm}
\usepackage{newtxtext}% 英数字フォントの設定を変更しないでください
\usepackage[varg]{newtxmath}% % 英数字フォントの設定を変更しないでください
%\usepackage{amssymb}
%\usepackage{bm}

\setcounter{page}{1}

\field{}
\jtitle{}
\etitle{}
\authorlist{%
 \authorentry{}{}{}{}\MembershipNumber{}
 %\authorentry{和文著者名}{英文著者名}{会員種別}{所属ラベル}\MembershipNumber{}
 %\authorentry{和文著者名}{英文著者名}{会員種別}{所属ラベル}[現在の所属ラベル]\MembershipNumber{}
}
\affiliate[]{}{}
%\affiliate[所属ラベル]{和文所属}{英文所属}
%\paffiliate[]{}
%\paffiliate[現在の所属ラベル]{和文所属}
\jalcdoi{???????????}% ← このままにしておいてください

\begin{document}
\begin{abstract}
%和文あらまし 120字以内
\end{abstract}
\begin{keyword}
%和文キーワード 4〜5語
\end{keyword}
\begin{eabstract}
%英文アブストラクト 50 words
\end{eabstract}
\begin{ekeyword}
%英文キーワード
\end{ekeyword}
\maketitle

\section{まえがき}


\ack %% 謝辞

%\bibliographystyle{sieicej}
%\bibliography{myrefs}
\begin{thebibliography}{99}% 文献数が 10 未満の時 {9}
\bibitem{}
\end{thebibliography}

\appendix
\section{}

\end{document}



%% 4. 「技術研究報告」
\documentclass[technicalreport]{ieicej}
%\usepackage[dvipdfmx]{graphicx,xcolor}
%%\usepackage[dvips]{graphicx}
\usepackage[fleqn]{amsmath}
%\usepackage{amsthm}
\usepackage{newtxtext}% 英数字フォントの設定を変更しないでください
\usepackage[varg]{newtxmath}% % 英数字フォントの設定を変更しないでください
%\usepackage{amssymb}
%\usepackage{bm}

\tecrepyear{202X}% X を適宜書き換えてください
\jtitle{}
\jsubtitle{}
\etitle{}
\esubtitle{}
\authorlist{%
 \authorentry[]{}{}{}
% \authorentry[メールアドレス]{和文著者名}{英文著者名}{所属ラベル}
}
\affiliate[]{}{}
%\affiliate[所属ラベル]{和文勤務先\\ 連絡先住所}{英文勤務先\\ 英文連絡先住所}

%\MailAddress{$\dagger$hanako@denshi.ac.jp,
% $\dagger\dagger$\{taro,jiro\}@jouhou.co.jp}

\begin{document}
\begin{jabstract}
%和文あらまし
\end{jabstract}
\begin{jkeyword}
%和文キーワード
\end{jkeyword}
\begin{eabstract}
%英文アブストラクト
\end{eabstract}
\begin{ekeyword}
%英文キーワード
\end{ekeyword}
\maketitle

\section{はじめに}


%\bibliographystyle{sieicej}
%\bibliography{myrefs}
\begin{thebibliography}{99}% 文献数が10未満の時 {9}
\bibitem{}
\end{thebibliography}

\end{document}
